\usepackage[italian]{babel}
\usepackage[T1]{fontenc}
\usepackage[utf8]{inputenc} % si possono usare caratteri accentati direttamente
\usepackage{macros}

% --- FONT ---
\usepackage{lmodern}
% ------------

\usepackage{hyperref}
\usepackage{chngcntr} % \counterwithout
\usepackage[useregional]{datetime2} % (!) tenere dopo {babel}
\usepackage{fancyhdr}

% --- THEOREMS ---
\declaretheorem[style=definition,name=Definizione]{defn}
\declaretheorem[style=plain,name=Teorema]{teo}
\declaretheorem[style=plain,name=Proposizione]{prop}
\declaretheorem[style=remark,name=Esercizio]{ex}
\newenvironment{solution}
{\noindent\emph{Soluzione.}\footnotesize}
{}
% ----------------

\DTMlangsetup{showyear=false}

\newcommand\includelesson[4]{%
\begin{lesson}{#1}{#2}{#3}{#4}
\input{lez-#1-#2-#3}
\end{lesson}}

% --- SEARCH PATHS ---
\makeatletter
\providecommand*{\input@path}{}
\g@addto@macro\input@path{{figure/}, {lezioni/}}
\makeatother
% --------------------

%%% --- DOCUMENT --- %%%

\title{Appunti di Analisi statistica dei dati}
\author{%
Giacomo Petrillo}
\date{Anno accademico 2017--2018}

\begin{document}

\pagestyle{fancy}
\renewcommand{\headrulewidth}{0pt}

% \frontmatter

\fancyhf{}
\fancyhead[LO,RE]{\rightmark}
\fancyhead[RO,LE]{\thepage}

\maketitle

\tableofcontents
\clearpage

% \mainmatter

\fancyhf{}
\fancyhead[LO,RE]{\rightmark}
\fancyhead[RO]{{\small\headcmacro}\hspace{2cm}\thepage}
\fancyhead[LE]{\thepage\hspace{2cm}{\small\headcmacro}}

\chapter{Probabilità}

\includelesson{2017}{09}{20}{Morello}

\renewcommand*\listtheoremname{Definizioni, teoremi, esercizi}
\listoftheorems[onlynamed]

\end{document}
