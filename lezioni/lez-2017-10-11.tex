% Giacomo Petrillo
% lezione di Morello

\begin{exercise}
	Indichiamo con 1 le persone che hanno una certa malattia e con 0 quelle che non ce l'hanno.
	Un test medico per la malattia è una procedura che può dare risultato ``$+$'' o ``$-$'',
	dove si spera che $+$ indichi che la persona ha la malattia.
	Il test è caratterizzato una probabilità di falso positivo $P(+|0)$ e una di falso negativo
	$P(-|1) = 1 - P(+|1)$.
	Quello che interessa al paziente è la probabilità di avere la malattia in base al risultato del test:
	$P(1|\pm)$. Calcolarlo conoscendo la frazione di popolazione che ha la malattia $P(1)$, anche numericamente con i valori: $P(+|0)=0.03$, $P(-|1)=0.02$, $P(1)=10^{-3}$.
\end{exercise}

\begin{solution*}
	È una semplice applicazione del teorema di Bayes:
	\begin{align*}
		P(1|\pm) &= \frac{P(\pm|1) P(1)}{P(\pm)} \\
		\Big[ P(\pm) &= P(\pm|1)P(1) + P(\pm|0)P(0) \Big] \\
	\end{align*}
	che sostituendo i valori numerici risulta:
	\begin{align*}
		P(1|+) &= 0.032 \\
		P(1|-) &= \num{2.1e-5}.
	\end{align*}
\end{solution*}

\begin{exercise}
	Un rivelatore di particelle è composto da due contatori.
	Al passaggio di una particella,
	un contatore ha probabilità $p$ o $q$ di inviare un segnale se la particella è rispettivamente
	un elettrone ($e$) o un fotone ($\gamma$).
	Gli eventi possibili sono 0, 1, 2 cioè il numero di contatori che ha inviato un segnale.
	Calcolare $P(\gamma|1)$ e $P(\gamma|2)$ per un fascio di particelle con frazione di elettroni $P(e)$,
	anche con i valori numerici $p=0.97$, $q=10^{-3}$, $P(e)=10^{-4}$.
\end{exercise}

\begin{solution*}
	È anche questa una semplice applicazione di Bayes.
	\begin{align*}
		P(\gamma|12) &= \frac{P(12|\gamma) P(\gamma)}{P(12)} \\
		P(12) &= P(12|\gamma)P(\gamma) + P(12|e)P(e) \\
		P(1|\gamma) &= 2q(1-q), \quad P(1|e) = 2p(1-p) \\
		P(2|\gamma) &= q^2, \quad P(2|e) = p^2 \\
		P(\gamma) &= 1 - P(e).
	\end{align*}
	Mettendo i numeri:
	\begin{align*}
		P(\gamma|1) &= 1 - \num{3e-3} \\
		P(\gamma|2) &= 0.01.
	\end{align*}
\end{solution*}

\begin{exercise}
	Abbiamo un deposito di munizioni stipate in casse.
	Da indagini precendenti sappiamo che ci sono casse ``malandate''
	in cui una frazione $q_0$ delle munizioni è difettosa,
	mentre nelle altre casse tutti i proiettili sono buoni.
	La probabilità che una cassa sia malandata è $f$.
	Vogliamo migliorare la qualità aprendo tutte le casse,
	controllando in ognuna una quantità fissata di munizioni e
	eliminando la cassa se troviamo una munizione difettosa.
	Calcolare $f$ in seguito a questa procedura.
\end{exercise}

\begin{solution*}
	Fissiamo la notazione:
	$q$ è la frazione di proiettili difettosi in un cassa,
	$n$ è il numero di munizioni che controlliamo in ogni cassa,
	$k$ è il numero di munizioni difettose tra quelle che controlliamo in una cassa.
	
	Quello che ci interessa è $P(q=0|k=0)$.
	Il priore su $q$ è
	\begin{equation*}
		P(q) = (1-f) \delta(q) + f \delta(q-q_0).
	\end{equation*}
	La distribuzione di $k$ è binomiale:
	\begin{equation*}
		P(k|q) = \binom nk q^k (1-q)^{n-k}.
	\end{equation*}
	La soluzione è ora immediata:
	\begin{align*}
		P(q|k) &= \frac{P(k|q) P(q)}{P(k)} \\
		\Bigg[ P(k) &= \int \de q\, P(k|q) P(q) \Bigg].
	\end{align*}
	Si può rendere più realistico l'esercizio usando un priore su $q$ non discreto.
\end{solution*}

\marginpar{qui manca un esercizio perché mi sono distratto
aveva a che fare con la temperatura e i taxi}

\begin{example}
	Consideriamo un modello poissoniano con priore \emph{improprio} uniforme:
	\begin{align*}
		P(k|\mu) &= \frac{\mu^k}{k!}e^{-\mu} \quad (\mu > 0) \\
		P(\mu) &= 1.
	\end{align*}
	Anche se il priore è non normalizzabile, il posteriore lo è:
	\begin{align*}
		P(\mu|k_0) &= \frac{P(k_0|\mu) P(\mu)}{P(k_0)} \\
		P(k_0) &= \int_0^\infty \de\mu\, \frac{\mu^{k_0}}{k_0!}e^{-\mu} = \frac{\Gamma(1+k_0)}{k_0!} = 1.
	\end{align*}
	Se cambiamo parametrizzazione a $\tau=1/\mu$ e scegliamo un priore uniforme su $\tau$,
	il posteriore non è più normalizzabile per $k_0<2$:
	\marginpar{non è per $k_0\le 1$?}%
	\begin{equation*}
		P(k_0)
		= \int_0^\infty \de\tau\, \frac{\tau^{-k_0}}{k_0!}e^{-1/\tau}
		= \frac1{k_0!} \int_0^\infty \frac{\de\mu}{\mu^2} \mu^{k_0}e^{-\mu}.
	\end{equation*}
\end{example}
