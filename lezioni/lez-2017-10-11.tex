% Giacomo Petrillo
% lezione di Morello

\begin{exercise}
	Indichiamo con 1 le persone che hanno una certa malattia e con 0 quelle che non ce l'hanno.
	Un test medico per la malattia è una procedura che può dare risultato ``$+$'' o ``$-$'',
	dove si spera che $+$ indichi che la persona ha la malattia.
	Il test è caratterizzato una probabilità di falso positivo $P(+|0)$ e una di falso negativo
	$P(-|1) = 1 - P(+|1)$.
	Quello che interessa al paziente è la probabilità di avere la malattia in base al risultato del test:
	$P(m|t)$. Calcolarlo conoscendo la frazione di popolazione che ha la malattia $P(m)$, anche numericamente con i valori: $P(+|0)=0.03$, $P(-|1)=0.02$, $P(1)=10^{-3}$.
\end{exercise}

\begin{solution*}
	È una semplice applicazione del teorema di Bayes:
	\begin{align*}
		P()
	\end{align*}
\end{solution*}
