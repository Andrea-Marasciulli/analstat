% Giacomo Petrillo
% lezione di Punzi

\section{Stima intervallare}

Quando, riportando il risultato di un'inferenza, si scrive
\begin{equation*}
	\text{<<}\theta = a \pm b\text{>>},
\end{equation*}
convenzionalmente si intende che $a$ è una stima di $\theta$ con varianza che vale $b^2$ se calcolata in $a$,
e bias trascurabile rispetto a $b$.
Tipicamente si da anche per scontato che,
se non specificato esplicitamente,
la distribuzione dello stimatore sia approssimabile con una gaussiana.
A volte si riporta un terzo numero:
\begin{equation*}
	\text{<<}\theta = a \pm b \pm c\text{>>},
\end{equation*}
dove $c$ quantifica in qualche modo l'incertezza sistematica.
A volte si trovano incertezze \emph{asimmetriche}
\begin{equation*}
	\text{<<}\theta = a^{+b}_{-c}\text{>>}
\end{equation*}
che per adesso lasciamo da parte.

\begin{example}
	Consideriamo la poissoniana.
	Vogliamo stimare la media $\mu$ da un conteggio $k$.
	Sappiamo che il conteggio stesso è lo stimatore di massima likelihood,
	ha bias nullo ed è sufficiente quindi anche efficiente,
	allora lo usiamo senza indugio.
	La varianza di $k$ è $\mu$,
	quindi scriveremo
	\begin{equation*}
		\text{<<}\mu = k \pm \sqrt k\text{>>}.
	\end{equation*}
	Notiamo che, se $k=0$, otteniamo
	\begin{equation*}
		\text{<<}\mu = 0 \pm 0\text{>>};
	\end{equation*}
	questo modo di scrivere il risultato non ci piace perché sembra dire che $\mu$ è sicuramente nullo.
	In generale quello che vorremmo è una stima della varianza con una certa incertezza relativa,
	perché è uso riportare l'incertezza con un certo numero di cifre significative.
	L'incertezza relativa sulla stima della varianza è $\sqrt\mu / k$,
	quindi per $\mu$ piccolo (e quindi $k$ piccoli) la notazione ha un significato molto meno forte.
\end{example}

Detto più in generale,
quando la deviazione stadard dello stimatore varia sensibilmente
(ad esempio, se vogliamo riportare l'incertezza con due cifre, più dell'\SI1\%)
in un'intorno del valore vero di larghezza dell'ordine della deviazione standard stessa,
riportare la deviazione standard calcolata nel valore dello stimatore ottenuto è poco utile.
Sono stati studiati metodi ad hoc per riportare l'incertezza in questi casi,
ma non sono definitivamente soddisfacenti e comunque non sono applicabili in generale.
Passando al primo ordine, vorremmo che:
\begin{align*}
	\dv{\sigma_{\hat\theta}}{\theta} \cdot \sigma_{\hat\theta}
	&\ll \sigma_{\hat\theta} \implies \\
	\implies \dv{\sigma_{\hat\theta}}{\theta}
	&\ll 1.
\end{align*}
Un caso specifico in cui bisogna conoscere con precisione la varianza è la media pesata\footnote{Vedi \autoref{th:wavg}.}:
in questo caso la varianza entra direttamente del valore del nuovo stimatore.

C'è un altro problema con questa notazione,
in questo caso non intrinseco ma di errata interpretazione.
Al risultato
<<$\theta=a\pm b$>>
non è in generale associato in alcun modo l'intervallo $(a-b,a+b)$.
Purtroppo è d'uso riportare l'incertezza nei grafici proprio con una ``barra d'errore''
che raffigura questo intervallo.

La soluzione generale a questi problemi è la \emph{stima intervallare}.
Anziché usare uno stimatore a valori nello spazio del parametro,
si ottiene un elemento dell'insieme delle parti.
Ovviamente la stima intervallare non è limitata a intervalli monodimensionali,
tuttavia è il caso tipico.
Riassumiamo i motivi per preferire una stima intervallare piuttosto che una puntuale:
\begin{itemize}
	\item la varianza dello stimatore varia sensibilmente nella regione di interesse;
	\item la distribuzione dello stimatore è molto diversa da una gaussiana.
\end{itemize}

\subsection{Intervalli bayesiani}

Prima di trattare la stima intervallare,
vediamo concetti analoghi nel caso bayesiano,
che è più semplice.
Nell'analisi bayesiana il posteriore è il risultato dell'inferenza.
Tuttavia scrivere la formula di un posteriore non aiuta a capire il risultato;
è necessario in qualche modo visualizzare il posteriore.
Il minimo necessario è o calcolare la moda
(che in qualche modo corrisponde alla stima puntuale),
oppure calcolare la probabilità di una certa regione (che corrisponde alla stima intervallare).
\begin{definition}[Credibilità]
	\label{th:cred}
	Dato un modello per $x\in X$ con parametro $\theta\in\Theta$
	e una funzione $\fundef[f]X\pset(\Theta)$,
	la \emph{credibilità} di un dato $x_0$ rispetto a $f$ è
	\begin{equation*}
		\cred_f(x_0)
		\is \int_{f(x_0)} \de \theta\, p(\theta|x_0).
	\end{equation*}
\end{definition}
La questione è come scegliere $f$.
Il criterio comune è chiedere che la credibilità abbia almeno un certo valore,
chiamato \emph{livello di confidenza} ($\mathrm{CL}$).
\marginpar{Qui $\mathrm{CL}$ non sta per <<confidence level>>
bensì per <<credibility level>>;
l'ambiguità citata dopo è sul simbolo, non sul nome.}
Fissato $\mathrm{CL}$, scegliere una $f$ che abbia $\cred_f(x_0)\ge\mathrm{CL}$
equivale a scegliere una \emph{funzione di ordinamento} $O(\theta,x)$ e porre
\begin{equation*}
	f(x) \is \{\theta\,|\,O(\theta,x)>q(\mathrm{CL})\},
\end{equation*}
dove $q(\mathrm{CL})$ è il massimo $q$ che soddisfi
\begin{equation*}
	\int_{O(\theta,x)>q} \de\theta\, p(\theta|x) \ge \mathrm{CL}.
\end{equation*}
Ad esempio se scegliamo $O(\theta,x)=\theta$ (e $\theta\in\R$),
otterremo la semiretta che comincia più a destra possibile la cui probabilità è almeno $\mathrm{CL}$.
Un modo intuitivo di scegliere la funzione di ordinamento è il \emph{posterior ordering}:
$O(\theta,x) = p(\theta|x)$.
Se $\theta\in\R$, il posterior ordering dà l'intervallo più corto possibile.
Tuttavia notiamo che se cambiamo parametro l'intervallo del posterior ordering trasforma in modo non banale,
quindi quale sia l'intervallo più ``corto'' è in realtà un concetto vago.

\begin{example}
	\label{th:poissbayes}
	Consideriamo la poissoniana.
	Con un priore improprio uniforme sulla media, si ottiene
	\begin{equation*}
		p(\mu|k=0)
		= e^{-\mu}.
	\end{equation*}
	Usiamo il posterior ordering:
	$e^{-\mu}$ è decrescente quindi trovare un valore minimo equivale a trovare un $\mu$ massimo:
	\begin{align*}
		\mathrm{CL}
		&= \int_0^{\mu_\text{max}} \de\mu\, e^{-\mu} = \\
		&= 1 - e^{-\mu_\text{max}} \implies \\
		\implies \mu_\text{max}
		&= -\log(1-\mathrm{CL}).
	\end{align*}
	Per $\mathrm{CL}=\SI{90}\%$,
	otteniamo $\mu_\text{max}=2.3$.
	Allora scriveremo il risultato come
	\begin{equation*}
		\text{<<$\mu < 2.3$ (\SI{90}\%)>>.}
	\end{equation*}
	Ricordiamo che con la stima puntuale $k$ avremmo ottenuto
	\begin{equation*}
		\text{<<}\mu = 0 \pm 0\text{>>.}
	\end{equation*}
\end{example}

\begin{example}
	Consideriamo la uniforme $p(x|m)=m^{-1}\chi_{(0,m)}(x)$.
	Prendiamo un priore improprio uniforme per $m$;
	con almeno due estrazioni il posteriore è normalizzabile, e si ottiene:
	\begin{equation*}
		p(m|\mathbf x)
		= \frac{(\max\mathbf x)^{N-1}}{(N-1)m^N}\chi_{(\max\mathbf x,\infty)}(m).
	\end{equation*}
	Usiamo il posterior ordering:
	\begin{align*}
		\mathrm{CL}
		&= \int_{\max\mathbf x}^{m_\mathrm{max}} \de m\,p(m|\mathbf x) = \\
		&= (\max\mathbf x)^{N-1}
		\left( \frac1{(\max\mathbf x)^{N-1}} - \frac1{m_\mathrm{max}^{N-1}} \right) \implies \\
		\implies m_\mathrm{max}
		&= \frac {\max\mathbf x} {\sqrt[N-1]{1-\mathrm{CL}}}.
	\end{align*}
	Con due estrazioni e $\mathrm{CL}=\SI{90}\%$ si ottiene $m_\mathrm{max}=10\cdot\max\{x_1,x_2\}$.
\end{example}
