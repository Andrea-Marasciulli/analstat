% Giacomo Petrillo
% lezione di Punzi

\begin{example}
	Consideriamo la pdf uniforme
	\begin{equation*}
		p(x;m) = \begin{cases}
			\frac1m & x\in(0,m) \\
			0 & \text{altrimenti}
		\end{cases}
		= \frac{\chi_{(0,m)}(x)}m.
	\end{equation*}
	Date $N$ estrazioni di $x$ calcoliamo la pdf del massimo
	\begin{equation*}
		X(\mathbf x) \is \max\{x_1,\dotsc,x_N\}.
	\end{equation*}
	Partiamo calcolando la cdf:
	\begin{align*}
		F(x;m) &= \begin{cases}
			0 & x \le 0 \\
			\frac xm & 0 < x < m \\
			1 & x \ge m
		\end{cases} \\
		P(X\le X_0)
		&= P(\forall i:x_i\le X_0) = \\
		&= P(x \le X_0)^N \\
		\implies
		p(X;m) &= \frac Nm \left( \frac Xm \right)^{N-1} \chi_{(0,m)}(X).
	\end{align*}
	Notiamo che la pdf delle estrazioni fattorizza con quella di $X$:
	\begin{align*}
		p(\mathbf x;m)
		&= \frac{\prod_{i=1}^N \chi_{(0,m)}(x_i)}{m^N} = \\
		&= \frac{\chi_{(0,\infty)}(\min(\mathbf x))\chi_{(-\infty,m)}(X(\mathbf x))}{m^N} = 
	\end{align*}
\end{example}
