% Giacomo Petrillo
% lezione di Punzi

\begin{example}
	\label{th:unifmaxpdf}
	Consideriamo la pdf uniforme
	\begin{equation*}
		p(x;m) = \begin{cases}
			\frac1m & x\in(0,m) \\
			0 & \text{altrimenti}
		\end{cases}
		= \frac{\chi_{(0,m)}(x)}m.
	\end{equation*}
	Date $N$ estrazioni di $x$ calcoliamo la pdf del massimo
	\begin{equation*}
		X(\mathbf x) \is \max\{x_1,\dotsc,x_N\}.
	\end{equation*}
	Partiamo calcolando la cdf:
	\begin{align*}
		F(x;m) &= \begin{cases}
			0 & x \le 0 \\
			\frac xm & 0 < x < m \\
			1 & x \ge m
		\end{cases} \\
		P(X\le X_0)
		&= P(\forall i:x_i\le X_0) = \\
		&= P(x \le X_0)^N \\
		\implies
		p(X;m) &= \frac Nm \left( \frac Xm \right)^{N-1} \chi_{(0,m)}(X).
	\end{align*}
	Notiamo che la pdf delle estrazioni fattorizza con quella di $X$:
	\begin{align*}
		p(\mathbf x;m)
		&= \frac{\prod_{i=1}^N \chi_{(0,m)}(x_i)}{m^N} = \\
		&= \frac{\chi_{(0,\infty)}(\min(\mathbf x))\chi_{(-\infty,m)}(X(\mathbf x))}{m^N} = \\
		&= \frac{\chi_{(0,\infty)}(\min(\mathbf x))}{N X(\mathbf x)^{N-1}}
		\frac Nm \left( \frac{X(\mathbf x)}m \right)^{N-1} \chi_{(0,m)}(X(\mathbf x)) = \\
		&= f(\mathbf x) p(X(\mathbf x);m),
	\end{align*}
	quindi il massimo è una statistica sufficiente per $m$.
\end{example}

\section{Informazione di Fisher}

\begin{definition}[Informazione di Fisher]
	Dato un modello per $x$ con parametro $\boldsymbol\theta\in\R^n$,
	l'\emph{informazione di Fisher} è la matrice
	\begin{equation*}
		I_{ij}(\boldsymbol\theta) \is E \left[
		\frac{\partial\log p(x;\boldsymbol\theta)}{\partial\theta_i}
		\frac{\partial\log p(x;\boldsymbol\theta)}{\partial\theta_j};\boldsymbol\theta \right].
	\end{equation*}
\end{definition}

Con la notazione <<$E[f;\theta]$>> intendiamo che
il valore atteso è calcolato con la probabilità corrispondente a $\theta$:
\begin{equation*}
	E[f;\theta] = \int \de x\, f(x) p(x;\theta).
\end{equation*}

Poiché l'informazione di Fisher richiede di calcolare il logaritmo della probabilità,
se ci sono degli $x$ per i quali la probabilità è nulla
è necessario restringere il dominio.
Espandiamo la definizione di $I$, abbreviando $p(x;\boldsymbol\theta)\isis L$:
\begin{align*}
	I_{ij}(\boldsymbol\theta)
	&= \int\de x\, L
	\frac{\partial\log L}{\partial\theta_i}
	\frac{\partial\log L}{\partial\theta_j} = \\
	&= \int\de x\,
	\frac{\partial\log L}{\partial\theta_i}
	\frac{\partial L}{\partial\theta_j} = \\
	&= \int\de x \left(
	\frac{\partial}{\partial\theta_j} \left( L \frac{\partial\log L}{\partial\theta_i} \right)
	- L \frac{\partial^2\log L}{\partial\theta_i\partial\theta_j} \right) = \\
	\intertext{(nei prossimi due passaggi portiamo fuori una derivata dall'integrale supponendo che il dominio non dipenda dal parametro)}
	&= \frac{\partial}{\partial\theta_j} \int\de x\, L \frac{\partial\log L}{\partial\theta_i}
	- \int\de x\, L \frac{\partial^2\log L}{\partial\theta_i\partial\theta_j} = \\
	&= \frac{\partial^2}{\partial\theta_i\partial\theta_j} \int\de x\, L
	- E \left[ \frac{\partial^2\log L}{\partial\theta_i\partial\theta_j} \right] = \\
	&= E \left[ -\frac{\partial^2\log p(x;\boldsymbol\theta)}{\partial\theta_i\partial\theta_j};\boldsymbol\theta \right].
\end{align*}
Scritta $I(\boldsymbol\theta)$ in questa forma,
si vede subito che se combino due variabili indipendenti le informazioni di Fisher si sommano.

\begin{example}
	Consideriamo $N$ estrazioni dell'esponenziale e calcoliamo l'informazione di Fisher:
	\begin{align*}
		p(\mathbf t;\lambda)
		&= \lambda^N e^{-\lambda\sum_{i=1}^N t_i}
		= \lambda^N e^{-\lambda N \bar t}, \quad \mathbf t\in(0,\infty)^N \\
		\log p(\mathbf t;\lambda)
		&= N\log\lambda - \lambda N\bar t \\
		\frac{\partial \log p(\mathbf t;\lambda)}{\partial\lambda}
		&= N \left( \frac1\lambda - \bar t \right) \\
		-\frac{\partial^2 \log p(\mathbf t;\lambda)}{\partial\lambda^2}
		&= \frac N{\lambda^2}.
	\end{align*}
	Come prima detto, combinando $N$ esponenziali indipendenti ottengo la somma delle informazioni delle singole esponenziali.
\end{example}

\begin{example}
	Consideriamo $N$ estrazioni della uniforme e calcoliamo l'informazione di Fisher sia con la definizione che con la formula con la derivata seconda:
	\begin{align*}
		p(\mathbf x;m)
		&= \frac1{m^N}, \quad \mathbf x\in(0,m)^N \\
		\log p(\mathbf x;m)
		&= -N\log m \\
		\left( \frac{\partial\log p(\mathbf x;m)}{\partial m} \right)^2
		&= \frac{N^2}{m^2} \\
		-\frac{\partial^2\log p(\mathbf x;m)}{\partial m^2}
		&= -\frac N{m^2}.
	\end{align*}
	Il conto con la seconda formula dà un risultato sbagliato, infatti il dominio dipende da~$m$.
\end{example}

\begin{example}
	\marginpar{Nella sezione della gaussiana, definire la gaussiana multidimensionale partendo dal prodotto di quella monodimensionale e facendo una trasformazione lineare dello spazio.}
	Consideriamo una gaussiana multidimensionale con matrice di covarianza fissata $V$ (che è necessariamente simmetrica e definita positiva), e calcoliamo l'informazione di Fisher per la media:
	\begin{align*}
		p(\mathbf x;\boldsymbol \mu)
		&= \frac1{\sqrt{\det(2\pi V)}}
		\exp \left( -\frac12 (\mathbf x-\boldsymbol\mu)^\transp V^{-1} (\mathbf x-\boldsymbol\mu) \right) \\
		\log p(\mathbf x;\boldsymbol\mu)
		&= -\frac12\log\det(2\pi V)
		- \frac12 (\mathbf x-\boldsymbol\mu)^\transp V^{-1} (\mathbf x-\boldsymbol\mu) = \\
		&= (\ldots) - \frac12 (x_i-\mu_i)V^{-1}_{ij}(x_j-\mu_j) \\
		\frac{\partial\log p(\mathbf x;\boldsymbol\mu)}{\partial\mu_k}
		&= -\frac12 V^{-1}_{ij} \big( -\delta_{ik}(x_j-\mu_j) - \delta_{jk}(x_i-\mu_i) \big) = \\
		\intertext{(l'inversa di una matrice simmetrica è simmetrica)}
		&= V^{-1}_{ik}(x_i-\mu_i) \\
		-\frac{\partial^2\log p(\mathbf x;\boldsymbol\mu)}{\partial\mu_l\partial\mu_k}
		&= -V^{-1}_{ik}(-\delta_{il}) = V^{-1}_{lk} \\
		\implies I(\boldsymbol\mu) &= V^{-1}.
	\end{align*}
\end{example}

Consideriamo una statistica sufficiente:
\begin{align*}
	p(x;\theta)
	&= f(x)p(s(x);\theta) \\
	\frac{\partial}{\partial\theta}\log p(x;\theta)
	&= \frac{\partial}{\partial\theta} \big( \log f(x) + \log p(s(x);\theta) \big) = \\
	&= \frac{\partial}{\partial\theta}\log p(s(x);\theta),
\end{align*}
quindi l'informazione di Fisher di una statistica sufficiente è uguale a quella originale.

\begin{fact}
	\label{th:fishstat}
	L'informazione di Fisher di una statistica è minore o uguale all'originale.
\end{fact}

\noindent In generale il fatto che l'informazione di Fisher sia uguale non implica che la statistica sia sufficiente, tuttavia per vari esempi comuni è vero.

\begin{example}
	Calcoliamo l'informazione di Fisher di un istogramma ordinato di un'esponenziale.
	Sia $\Delta$ la larghezza dei bin.
	Per ogni estrazione, la probabilità di cadere nel $k$-esimo bin ($k\ge 0$) è
	\begin{align*}
		p(k;\lambda)
		&= \int_{k\Delta}^{(k+1)\Delta}\de t\, \lambda e^{-\lambda t} = \\
		&= \big[-e^{-\lambda t}\big]_{k\Delta}^{(k+1)\Delta} = \\
		&= e^{-\lambda k\Delta}(1-e^{-\lambda\Delta}).
	\end{align*}
	L'istogramma ordinato è la $N$-upla $\mathbf k$ dei bin in cui cadono $N$ estrazioni.
	Calcoliamo l'informazione di Fisher per $N=1$.
	\begin{align*}
		\frac{\partial}{\partial\lambda} \log p(k;\lambda)
		&= -k\Delta + \frac{\Delta e^{-\lambda\Delta}}{1-e^{-\lambda\Delta}} \\
		-\frac{\partial^2}{\partial\lambda^2} \log p(k;\lambda)
		&= \frac{\partial^2}{\partial\lambda^2}
		\left( \frac{\Delta}{1-e^{\lambda\Delta}} \right) = \\
		&= \frac {\Delta^2 e^{\lambda\Delta}} {(e^{\lambda\Delta}-1)^2} = \\
		&= \frac1{\lambda^2} \left( 1 - \frac{\lambda^2\Delta^2}{12} + o(\lambda^2\Delta^2) \right)
	\end{align*}
	Il caso generale si ottiene moltiplicando per $N$.
	Notiamo che nel limite $\Delta\to0$ riotteniamo l'informazione dell'esponenziale.
\end{example}
