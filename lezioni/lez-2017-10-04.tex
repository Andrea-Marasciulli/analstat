% Giacomo Petrillo
% lezione di Francavilla

\begin{figure}
	\centering
	\includegraphics[width=9cm]{binomiale}
	\caption{Distribuzione binomiale per $n=5$ e $p=0.5, 0.8$.}
\end{figure}

\begin{exercise}
	Calcolare direttamente media e varianza della binomiale.
\end{exercise}

\begin{solution}
	\begin{align*}
		E[k] &=
		\sum_{k=0}^n k \binom nk p^k (1-p)^{n-k} = \\
		\Bigg[ k \binom nk &= k \frac{n!}{(n-k)!k!} = \frac{n!(n-k+1)}{(n-k+1)!(k-1)!} = (n-k+1) \binom n{k-1} \Bigg] \\
		&= (n+1) \sum_{k=1}^n \binom n{k-1} p^k (1-p)^{n-k} + {} \\
		&\phantom{{}={}} - \sum_{k=1}^n k \binom n{k-1} p^k (1-p)^{n-k} = \\
		&= (n+1) \sum_{k=0}^{n-1} \binom nk p^{k+1} (1-p)^{n-k+1} + {} \\
		&\phantom{{}={}} - \sum_{k=0}^{n-1} (1+k) \binom nk p^{k+1} (1-p)^{n-k+1} = \\
		&= \frac p{1-p} \big( (n+1)(E[1] - p^n) - (E[1] + E[k] - (1+n)p^n) \big) = \\
		&= \frac p{1-p} ( n - E[k] ) \rightarrow \\
		&\rightarrow E[k]\left(1+\frac p{1-p} \right) = \frac {np}{1-p} \rightarrow \\
		&\rightarrow E[k] = np
	\end{align*}
	\begin{align*}
		E[k^2] &=
		\sum_{k=0}^n k^2 \binom nk p^k (1-p)^{n-k} = \\
		&= \sum_{k=0}^n k(n+1-k) \binom n{k-1} p^k (1-p)^{n-k} = \\
		&= (n+1) \sum_{k=1}^n k \binom n{k-1} p^k (1-p)^{n-k} + {} \\
		&\phantom{{}={}} - \sum_{k=1}^n k^2 \binom n{k-1} p^k (1-p)^{n-k} = \\
		&= (n+1) \sum_{k=0}^{n-1} (1+k) \binom nk p^{k+1} (1-p)^{n-k-1} + {} \\
		&\phantom{{}={}} - \sum_{k=0}^{n-1} (1+k)^2 \binom nk p^{k+1} (1-p)^{n-k-1} = \\
		&= \frac p{1-p} \Big(
		(n+1)(E[1] + E[k] - (n+1)p^n) + {} \\
		&\phantom{{} = \frac p{1-p} \big({}} - (E[1] + 2E[k] + E[k^2] - (n+1)^2p^n) \Big) \rightarrow \\
		\rightarrow E[k^2] &= p\big((n+1)(1+np) - 1 - 2np\big) = \\
		&= -np^2 + np + n^2p^2
	\end{align*}
	\begin{equation*}
		\sigma^2 = E[k^2] - E[k]^2 = np(1-p)
	\end{equation*}
\end{solution}

Noi la calcoliamo invece con la funzione generatrice:
\begin{align*}
	\frac{\de M}{\de t} &=
	n(pe^t + 1 - p)^{n-1}pe^t \\
	\frac{\de^2 M}{\de t^2} &=
	np \big(e^t(pe^t + 1-p)^{n-1} + e^t(n-1)(pe^t+1-p)^{n-2}pe^t\big) = \\
	&= npe^t(pe^t + 1-p)^{n-2}(pe^t+1-p + pe^t(n-1)) \\
	\mu &= \left. \frac{\de M}{\de t} \right|_{t=0} = np \\
	\mu_2 &= \left. \frac{\de^2 M}{\de t^2} \right|_{t=0} = np(1 + p(n-1)) = np + n^2p^2 - np^2 \\
	\sigma^2 &= \mu_2 - \mu^2 = np(1-p).
\end{align*}

\begin{exercise}
	Calcolare la distribuzione del numero $n$ di lanci necessari per ottenere $k$ successi.
\end{exercise}

\begin{solution}
	Dobbiamo contare le sequenze binarie lunghe $n$ che hanno $k$ uno e finiscono per uno,
	cioè contare le sequenze lunghe $n-1$ con $k-1$ uno.
	La probabilità di una sequenza è $p^k(1-p)^{n-k}$.
	Quindi
	\begin{equation*}
		P(n;k) \propto \binom{n-1}{k-1} (1-p)^n.
	\end{equation*}
	La normalizzazione è
	\begin{equation*}
		\sum_{n=k}^\infty \binom{n-1}{k-1} (1-p)^n =
		\sum_{n=k}^\infty O(n^k) (1-p)^n
	\end{equation*}
	che converge.
	Nota: non è corretto il ragionamento
	<<$P(k;n)$ è la binomiale, ricavo $P(n;k)$ usando Bayes>>
	perché qui sappiamo che al $k$-esimo successo smettiamo di lanciare.
\end{solution}

\begin{exercise}
	Calcolare media e varianza della Poissoniana.
\end{exercise}

\begin{solution*}
	Calcoliamo la generatrice
	\begin{align*}
		M(t) &= E[e^{tk}] = \\
		&= \sum_k e^{tk} \frac{\mu^k}{k!} e^{-\mu} =
		e^{-\mu} \sum_k \frac{(\mu e^t)^k}{k!} = \\
		&= e^{-\mu} e^{\mu e^t} = e^{\mu(e^t - 1)}.
	\end{align*}
	Dalla generatrice ricaviamo la media e la varianza
	\begin{align*}
		\frac{\de M}{\de t} &= e^{\mu(e^t-1)}\mu e^t = \mu e^{\mu(e^t-1) + t} \\
		\frac{\de^2 M}{\de t^2} &= \mu e^{\mu(e^t-1) + t} (\mu e^t + 1) \\
		E[k] &= \left. \frac{\de M}{\de t} \right|_{t=0} = \mu \\
		\var[k] &= \left. \frac{\de^2 M}{\de t^2} \right|_{t=0} - E[k]^2 = \mu.
	\end{align*}
\end{solution*}

\begin{figure}
	\centering
	\includegraphics[width=9cm]{poisson}
	\caption{Distribuzione poissoniana per $\mu=2,5$.}
\end{figure}

\begin{exercise}
	Distribuzione della somma di variabili poissoniane.
\end{exercise}

\begin{solution}
	Si ricava immediatamente dalla funzione generatrice:
	\begin{equation*}
		M_{k_1+k_2}(t) =
		M_{k_1}(t)M_{k_2}(t) =
		e^{\mu_1(e^t-1)} e^{\mu_2(e^t-1)} =
		e^{(\mu_1+\mu_2)(e^t-1)}
	\end{equation*}
	quindi la distribuzione è ancora poissoniana con $\mu=\mu_1+\mu_2$.
	Notiamo che si arriva alla stessa conclusione partendo dal fatto che la poissoniana è la distribuzione del numero di eventi in un certo tempo $T$ con una definita probabilità per unità di tempo $\lambda$; poiché quello che conta è il prodotto $\lambda T$ possiamo prendere $\lambda_1=\lambda_2$ quindi sommare due conteggi di eventi è come concatenare due intervalli di tempo.
\end{solution}

\begin{exercise}
	Per un certo tempo osservo la strada e conto i taxi,
	tuttavia ho una probabilità $1-\epsilon$ di distrarmi e non contare un taxi.
	Qual è la distribuzione del mio conteggio?
\end{exercise}

\begin{solution*}
	Assumiamo che il flusso medio di taxi non cambi nell'intervallo di osservazione,
	quindi il numero di taxi che passano è una poissoniana con media~$\mu$:
	\begin{equation*}
		P(n;\mu) = \frac{\mu^n}{n!}e^{-\mu}.
	\end{equation*}
	Dei taxi che passano, quelli che effettivamente conto è una binomiale:
	\begin{equation*}
		P(k|n;\epsilon) = \binom nk \epsilon^k (1-\epsilon)^{n-k}.
	\end{equation*}
	La probabilità di $k$ è allora
	\begin{align*}
		P(k;\mu,\epsilon) &=
		\sum_n P(k|n;\epsilon) P(n;\mu) = \\
		&= \sum_{n=k}^\infty \binom nk \epsilon^k (1-\epsilon)^{n-k} \frac{\mu^n}{n!}e^{-\mu} = \\
		&= \left( \frac\epsilon{1-\epsilon} \right)^k e^{-\mu} \sum_{n=k}^\infty \binom nk (1-\epsilon)^n \frac{\mu^n}{n!} = \\
		&= \left( \frac\epsilon{1-\epsilon} \right)^k \frac{e^{-\mu}}{k!} \sum_{n=k}^\infty \frac{(\mu(1-\epsilon))^n}{(n-k)!} = \\
		&= \left( \frac\epsilon{1-\epsilon} \right)^k \frac{e^{-\mu}}{k!} (\mu(1-\epsilon))^k e^{\mu(1-\epsilon)} = \\
		&= \frac{(\mu\epsilon)^k}{k!}e^{-\mu\epsilon},
	\end{align*}
	cioè poissoniana con media $\mu\epsilon$.
\end{solution*}

\begin{exercise}
	Media e varianza della distribuzione esponenziale.
\end{exercise}

\begin{solution*}
	La distribuzione esponenziale è
	\begin{equation*}
		p(x) = e^{-x}, \quad x \ge 0;
	\end{equation*}
	calcoliamo la funzione caratteristica:
	\begin{align*}
		\phi(t) &=  E[e^{itx}] = \\
		&= \int_0^\infty \de x\, e^{itx}e^{-x} = \int_0^\infty \de x\, e^{(it-1)x} = \\
		&= \left[ \frac{e^{(it-1)x}}{it-1} \right]_0^\infty = \frac1{1-it}.
	\end{align*}
	Calcoliamo i momenti:
	\begin{align*}
		\phi'(t) &= \frac{i}{(1-it)^2} \\
		\phi''(t) &= \frac{-2}{(1-it)^3} \\
		E[x] &= \frac1i \phi'(0) = 1 \\
		\var[x] &= \frac1{i^2}\phi''(0) - E[x]^2 = 2 - 1 = 1.
	\end{align*}
	Media e varianza per l'esponenziale nella forma $\lambda e^{-\lambda x}$ si ricavano immediatamente con un cambio di variabili:
	\begin{equation*}
		E[x] = \lambda^{-1}, \quad \var[x] = \lambda^{-2}.
	\end{equation*}
\end{solution*}

\begin{exercise}
	Ho un rate di eventi alto abbastanza da non poterli salvare tutti. Posso scegliere di salvare un evento ogni $n$ oppure di decidere casualmente per ogni evento di salvarlo con probabilità $1/n$. La distribuzione temporale degli eventi salvati è la stessa nei due casi?
\end{exercise}

\begin{solution}
	Il tempo che intercorre tra un evento e quello successivo è distribuito esponenzialmente:
	\begin{equation*}
		p(t) = e^{-t};
	\end{equation*}
	quindi il tempo che intercorre tra $n$ eventi è la somma di $n$ variabili esponenziali.
	Calcoliamo la distribuzione usando la funzione caratteristica:
	\begin{align*}
		\phi_t(u) &= \frac1{1-iu} \\
		\phi_{\sum_{i=1}^n t_i}(u) &= \frac1{(1-iu)^n} \\
		p_{\sum_{i=1}^n t_i}(t)
		&= \frac1{2\pi} \int_{-\infty}^\infty \de u\, e^{-itu} \frac1{(1-iu)^n} = \\
		&= \frac{e^{-t}}{2\pi} \int_{-\infty}^\infty \de u\, \frac{e^{(1-iu)t}}{(1-iu)^n} = \\
		&= \frac{e^{-t}}{2\pi}
		\left[ \frac{-i}{n-1} \frac1{(1-iu)^{n-1}} e^{(1-iu)t} \right]_{-\infty}^\infty + & \Bigg\} =0\\
		&\phantom{{}={}} - \frac{e^{-t}}{2\pi}
		\int_{-\infty}^\infty \de u\, \frac{-i}{n-1} \frac1{(1-iu)^{n-1}} (-it) e^{(1-iu)t} = \\
		&= \frac{e^{-t}}{2\pi} \frac{t}{n-1} \int_{-\infty}^\infty \de u\, \frac{e^{(1-iu)t}}{(1-iu)^{n-1}} = \\
		&= \frac{e^{-t}}{2\pi} \frac{t^{n-1}}{(n-1)!} \int_{-\infty}^\infty \de u\, \frac{e^{(1-iu)t}}{1-iu} = \\
		&= \frac{e^{-t}}{2\pi} \frac{t^{n-1}}{(n-1)!} 2\pi e^t \frac1{2\pi} \int_{-\infty}^\infty \de u\, e^{-itu} \phi_t(u) = \\
		&= \frac{t^{n-1}}{(n-1)!} e^{-t}
	\end{align*}
	Fissato un evento salvato, la probabilità che quello salvato successivo sia il $k$-esimo successivo è la probabilità di salvare quell'evento per la probabilità di non aver salvato quelli precedenti, cioè
	\begin{equation*}
		P(k) = p(1-p)^{k-1}, \quad p\is\frac1n.
	\end{equation*}
	La distribuzione del tempo fino al $k$-esimo evento è quella calcolata prima, allora
	\begin{align*}
		p(t) &= \sum_{k=1}^\infty p(t|k) P(k) = \\
		&= \sum_{k=1}^\infty \frac{t^{k-1}}{(k-1)!}e^{-t} \,  p(1-p)^{k-1} = \\
		&= e^{-t}p \sum_{k=0}^\infty \frac{t^k}{k!}(1-p)^k = \\
		&= e^{-t}p e^{t(1-p)} = pe^{-pt},
	\end{align*}
	che è ancora esponenziale.
	In particolare quindi le due distribuzioni sono diverse.
\end{solution}

\begin{exercise}
	Funzione caratteristica della Cauchy.
\end{exercise}

\begin{solution}
	Dobbiamo calcolare
	\begin{equation*}
		\int_{-\infty}^\infty \de x\, e^{itx} \frac1\pi \frac1{1+x^2}.
	\end{equation*}
	L'integrando va a zero almeno come $1/x^2$ per $t><0$ in uno dei due semipiani complessi quindi usiamo i residui.
	I poli sono in $\pm i$, sviluppiamo:
	\begin{align*}
		&\phantom{{}={}} \frac1\pi \frac{e^{it(\pm i)}}{1+ (\pm i + \Delta x^2)} = \\
		&= \frac1\pi \frac{e^{\mp t}}{\pm 2i\Delta x + O(\Delta x^2)} \approx \\
		&\approx \pm\frac{e^{\mp t}}{2\pi i} \frac1{\Delta x}.
	\end{align*}
	Se $t><0$ dobbiamo usare il semipiano rispettivamente superiore/inferiore,
	nel secondo caso la retta reale viene percorsa in verso opposto quindi l'integrale è
	\begin{equation*}
		\pm 2\pi i \left ( \pm\frac{e^{\mp t}}{2\pi i} \right) = e^{-|t|}.
	\end{equation*}
\end{solution}
