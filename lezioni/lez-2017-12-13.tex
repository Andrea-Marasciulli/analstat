% Giacomo Petrillo
% lezione di Punzi

\subsection{Istogrammi}

Consideriamo un istogramma con bin $S_j$ e conteggi $k_j$
di $N$ estrazioni di una distribuzione $p_0(x)$,
con $N$ poissoniano.
I $k_j$ sono poissoniani con medie
\begin{equation*}
	\mu p_j = \mu \int_{S_j} \de x\, p_0(x)
\end{equation*}
dove $\mu$ è la media di $N$.
Qualunque distribuzione $p_0$ viene tradotta in un elenco di medie di poissoniane.
Scriviamo il likelihood ratio con l'ipotesi nulla delle medie ottenute da $p_0$
e come ipotesi alternative tutte le altre medie possibili,
cioè le medie ottenibili da tutte le distribuzioni possibili sullo stesso spazio di~$p_0$:
\begin{align*}
	\lambda_{p_0}(\mathbf x)
	&= 2\log \frac
	{\sup\limits_{\boldsymbol\mu} p(\mathbf k;\boldsymbol\mu)}
	{p(\mathbf k;\mu \mathbf p)} = \\
	&= 2\log \frac
	{\prod_j \frac {k_j^{k_j}} {k_j!} e^{-k_j}}
	{\prod_j \frac {(\mu p_j)^{k_j}} {k_j!} e^{-\mu p_j}} = \\
	&= 2\log \prod_j \left(\frac{k_j}{\mu p_j}\right)^{k_j} e^{-(k_j-\mu p_j)} = \\
	&= 2\sum_j \left( k_j\left(\log\frac{k_j}{\mu p_j}-1\right) + \mu p_j \right).
\end{align*}
Dunque abbiamo ridotto un test tra una distribuzione e tutte le altre ditribuzioni possibili
a un test su poissoniane con medie diverse.

\subsection{Massima likelihood}

Per la gaussiana,
il logaritmo della likelihood è, a meno di termini che non contano, il $\chi^2$,
che sappiamo essere una buona statistica per un test.
In analogia con questo caso,
si può pensare di usare in generale il massimo del logaritmo della likelihood
come statistica per un test di goodness of fit.
E invece non funziona bene.
Basta pensare che, cambiando variabile,
la likelihood cambia di un termine che dipende dalla variabile.

\begin{example}
	Prendiamo la distribuzione uniforme
	\begin{equation*}
		p(\mathbf x) = \frac1{m^N},
		\quad \mathbf x \in (0,m)^N.
	\end{equation*}
	Il massimo della likelihood è $m^{-N}$ che non dipende da $\mathbf x$,
	quindi è perfettamente inutile come statistica per un test.
\end{example}

\subsection{Kolmogorov-Smirnov}

Consideriamo una distribuzione $p(x)$ per $x\in\R$.
Costruiamo la \emph{cumulante empirica} 
\begin{equation*}
	S_{\mathbf x}(x)
	\is \frac {\#\setdef[x_i]{x_i\le x}} N.
\end{equation*}
Sia $F$ la cumulante di $p$.
Definiamo la statistica
\begin{equation*}
	D(\mathbf x)
	\is \sup|S_\mathbf x-F|
\end{equation*}
che, per la monotonicità di $F$, si riscrive come
\begin{equation*}
	D = \max_{i=1,\dotsc,N} \Set{\left|\frac iN-F(x_i)\right|, \left|\frac{i-1}N-F(x_i)\right|},
\end{equation*}
quindi è facile da calcolare.
Si dimostra che la distribuzione di $D$ è la stessa per ogni $N$ e per ogni $p$,
e ha survival function asintotica
\begin{equation*}
	\lim_{N\to\infty} P(\sqrt N D>z)
	= 2 \sum_{k=1}^\infty (-1)^{k+1} e^{-2k^2z^2}.
\end{equation*}
Il test fatto con la survival function di $D$ come p-value si chiama \emph{test di Kolmogorov-Smirnov}.
Poiché il test KS è molto usato,
facciamo una lista di controllo:
\begin{itemize}
	\item controllare se si sta usando la distribuzione asintotica o quella per $N$ finito;
	\item controllare che l'ipotesi sia semplice,
	cioè che $p(x)$ è fissata e non è stata ottenuta dagli $x_i$ con un fit;
	\item se si approssima $p(x)$ con un istogramma, controllare che sia abbastanza fine.
\end{itemize}
Si può anche usare $D$ per stimare un parametro minimizzandola,
però è stato studiato e tipicamente funziona peggio di massima likelihood.
