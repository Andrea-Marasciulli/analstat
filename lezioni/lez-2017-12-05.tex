% Giacomo Petrillo
% lezione di Punzi

\section{Goodness of fit}

Supponiamo di voler decidere se assegnare o no una certa distribuzione a un osservabile.
Con i metodi visti fin'ora,
per trattare statisticamente la questione dobbiamo fissare quali siano le distribuzioni alternative.
Il problema di decidere se un osservabile segue o no una distribuzione
senza fissare altre distribuzioni va sotto il nome di \emph{goodness of fit}.
Non c'è una soluzione definitiva,
perché rimane vero che per trattare il problema in modo fondato
bisogna fissare un modello cioè una famiglia di distribuzioni.
Tuttavia esistono vari metodi convenzionali che
\begin{itemize}
	\item trattano modelli standard molto ampi;
	\item riconducono una distribuzione qualunque a una distribuzione fissata;
	\item pur senza specificare quali siano le distribuzioni alternative,
	in pratica funzionano per le distribuzioni che avrebbe senso specificare.
\end{itemize}
È comunque opportuno che un test di goodness of fit
non sia l'unico risultato di un'analisi.

\subsection{p-value}

Per i test di goodness of fit si usano le stesse notazioni di un test d'ipotesi,
però è comune usare come statistica del test un \emph{p-value}.

\begin{definition}[p-value]
	Un \emph{p-value} è una statistica che,
	sotto l'ipotesi nulla,
	ha distribuzione uniforme in $(0,1)$.
\end{definition}

\noindent È anche comune che la regione critica del p-value sia $(0,\alpha)$.
Tipicamente si riporta non solo il risultato del test ma anche il p-value,
perché in questo modo è possibile combinare più test.
Per combinare due p-value bisogna costruire una statistica uniforme di due uniformi.
Un modo comune è prendere la cumulante del prodotto:
\begin{align*}
	P(p_1p_2<q)
	&= \int_{p_1p_2<q}\de p_1\,\de p_2 = \\
	&= \int_0^1\de p_1\, \int_0^{\min\Set{1,q/p_1}} \de p_2 = \\
	&= \int_0^q\de p_1\, \int_0^1 \de p_2
	+ \int_q^1 \de p_1\, \int_0^{q/p_1} \de p_2 = \\
	&= q + q\int_q^1 \frac{\de p_1}{p_1} = \\
	&= q(1-\log q),
\end{align*}
quindi il p-value combinato è $p_{12} = p_1p_2(1-\log(p_1p_2))$.
Il modo comune di combinare più di due p-value è prendere la survival function ($1-\text{cdf}$) di
\begin{equation*}
	x = -2 \sum_{i=1}^n \log p_i.
\end{equation*}
La distribuzione di $-2\log p_i$ è $\chi^2$ a 2 gradi libertà,
quindi la distribuzione di $x$ è $\chi^2$ a $2n$ gradi di libertà.
Notiamo che abbiamo assunto che i p-value siano indipendenti,
che è il caso se i test che combiniamo sono su estrazioni diverse.
Se combiniamo due test sugli stessi dati
bisogna considerare la distribuzione congiunta.

\subsection{Test del $\chi^2$}

Prendiamo un modello gaussiano con varianza fissata e medie arbitrarie
\begin{equation*}
	p(\mathbf x;\boldsymbol\mu)
	= \prod_{i=1}^N \frac1{\sqrt{2\pi}\sigma_i}
	\exp\left(-\frac12\left(\frac{x_i-\mu_i}{\sigma_i}\right)^2\right).
\end{equation*}
Calcoliamo il likelihood ratio:
\begin{align}
	\lambda_{\boldsymbol\mu}(\mathbf x)
	&= 2\log \frac
	{\sup\limits_{\boldsymbol\mu'} p(\mathbf x;\boldsymbol\mu')}
	{p(\mathbf x;\boldsymbol\mu)} = \notag \\
	&= \sum_{i=1}^N \left(\frac{x_i-\mu_i}{\sigma_i}\right)^2. \label{eq:chi2}
\end{align}
Questa statistica si chiama $\chi^2$,
e infatti\footnote{Vedi \autoref{th:lrgauss}.} ha la distribuzione $\chi^2$ a $N$ gradi di libertà.
Usare come p-value la survival function della \eqref{eq:chi2}
è a tutti gli effetti un test likelihood ratio,
e si chiama \emph{test del $\chi^2$}.
Poiché in molti casi comuni la distribuzione della \eqref{eq:chi2} tende alla distribuzione $\chi^2$
per $N\to\infty$ anche se il modello non è gaussiano,
il test funziona bene con ipotesi alternative molto più vaste che il modello gaussiano con cui l'abbiamo ricavato.

Prendiamo ora un modello gaussiano con varianza fissata e medie che dipendono dal parametro $\theta\in\R^d$:
\begin{equation*}
	p(\mathbf x;\theta)
	= \prod_{i=1}^N \frac1{\sqrt{2\pi}\sigma_i}
	\exp\left(-\frac12\left(\frac{x_i-\mu_i(\theta)}{\sigma_i}\right)^2\right).
\end{equation*}
Si può dimostrare che,
detta $\hat\theta$ la stima di massima likelihood,
l'espressione analoga alla \eqref{eq:chi2}
\begin{equation*}
	\sum_{i=1}^N \left( \frac {x_i-\mu_i(\hat\theta(\mathbf x))} {\sigma_i} \right)^2
\end{equation*}
\marginpar{Ha esattamente la distribuzione a $N-d$ gradi di libertà o è asintotico anche per la gaussiana?}%
ha la distribuzione del $\chi^2$ a $N-d$ gradi di libertà.
Anche in questo caso si parla di test del $\chi^2$.

\begin{exercise}
	Scrivere il likelihood ratio di un modello per $N$ poissoniane con medie arbitrarie,
	e verificare (eventualmente numericamente) se asintoticamente ha la distribuzione $\chi^2$.
\end{exercise}
