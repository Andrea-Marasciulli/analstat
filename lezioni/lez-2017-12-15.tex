% Giacomo Petrillo
% lezione di Punzi

Facciamo il test likelihood ratio per $N$ estrazioni di una gaussiana
con l'ipotesi nulla composta di varianza arbitraria e media fissata.
La distribuzione è
\begin{align*}
	p(\mathbf x;\mu,\sigma)
	&= \left(\frac1{\sqrt{2\pi}\sigma}\right)^N
	\exp \left( -\frac12 \sum_i \left(\frac{x_i-\mu}{\sigma}\right)^2 \right),
\end{align*}
il likelihood ratio è
\begin{align*}
	\lambda_\mu(\mathbf x)
	&= 2\log \frac
	{\sup\limits_{\mu',\sigma} p(\mathbf x;\mu',\sigma)}
	{\sup\limits_{\sigma} p(\mathbf x;\mu,\sigma)}.
\end{align*}
Per calcolare i limiti superiori,
sappiamo già che la stima di massima likelihood con media e varianza libere è
\begin{align*}
	\hat\mu
	&= \bar x, \\
	\hat\sigma^2
	&= \frac 1N \sum_i (x_i-\bar x)^2;
\end{align*}
mentre con media fissata è
\begin{align*}
	\hat\sigma_2^2
	&= \frac 1N \sum_i (x_i-\mu)^2.
\end{align*}
Calcoliamo il likelihood ratio:
\begin{align*}
	\lambda_\mu(\mathbf x)
	&= 2\log \frac
	{\left(\frac1{\sqrt{2\pi}\hat\sigma}\right)^N
	\exp \left( -\frac12 \sum_i \left(\frac{x_i-\bar x}{\hat\sigma}\right)^2 \right)}
	{\left(\frac1{\sqrt{2\pi}\hat\sigma_2}\right)^N
	\exp \left( -\frac12 \sum_i \left(\frac{x_i-\mu}{\hat\sigma_2}\right)^2 \right)} = \\
	&= N \log \frac{\hat\sigma_2^2}{\hat\sigma^2}
	+ \frac{\sum_i (x_i-\mu)^2}{\hat\sigma_2^2}
	- \frac{\sum_i (x_i-\bar x)^2}{\hat\sigma^2} = \\
	&= N \log \frac
	{\sum_i (x_i - \mu)^2}
	{\sum_i (x_i - \bar x)^2} \simeq \\
	\intertext{(a meno di una funzione crescente)}
	&\simeq \frac
	{\sum_i (x_i - \mu)^2}
	{\sum_i (x_i - \bar x)^2} = \\
	&= \frac
	{\sum_i (x_i - \bar x)^2 + \sum_i (\bar x-\mu)^2}
	{\sum_i (x_i - \bar x)^2} = \\
	&= 1 + N \frac {(\bar x-\mu)^2} {\sum_i (x_i-\bar x)^2} \simeq \\
	\intertext{(a meno di una funzione crescente)}
	&\simeq \frac{\bar x - \mu}{S/\sqrt N} \isis t,
\end{align*}
dove $S$ è la varianza campione unbiased
\begin{equation*}
	S(\mathbf x)
	\is \frac {\sum_i (x_i-\bar x)^2} {N-1}.
\end{equation*}
La $t$ si chiama \emph{$t$ di Student} e ha distribuzione
\begin{equation*}
	p(t;\nu)
	= \frac {\Gamma\left(\frac{\nu+1}2\right)} {\sqrt{\pi\nu}\, \Gamma\left(\frac\nu2\right)}
	\left(1+\frac{t^2}\nu\right)^{-(\nu+1)/2}
\end{equation*}
con $\nu=N-1$.
Notiamo che per $\nu=1$ è la Cauchy.
In generale sono definiti i momenti solo fino a $\mu_{\nu-1}$.
Per $\nu\to\infty$, la distribuzione diventa gaussiana.
