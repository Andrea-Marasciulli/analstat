% Giacomo Petrillo
% lezione di Punzi

Supponiamo di avere un libro che contiene un elenco di numeri.
Si possono considerare questi numeri come estrazioni di una variabile casuale?
Riformulando la domanda come
<<si può applicare la teoria della probabilità a quella sequenza di numeri?>>
appare più evidente che la risposta è che in definitiva è arbitrario e,
se si fissa un criterio,
che dovrebbe essere lo stesso che useremmo per delle prove in un esperimento,
ad esempio per stabilire che l'uscita di un dado è ``casuale''.

Una sequenza prestabilita di numeri che consideriamo casuali,
detti \emph{pseudocasuali},
può essere utile innanzitutto per fare simulazioni di fenomeni che consideriamo casuali,
ma anche per fare calcoli numerici.
Se dimostriamo che una certa statistica converge in probabilità a una quantità che vogliamo calcolare,
possiamo raggiungere la precisione di calcolo desiderata aumentando la lunghezza della sequenza pseudocasuale.
I metodi di calcolo e simulazione basati su sequenze pseudocasuali si chiamano \emph{Monte Carlo}.

Supponiamo di voler calcolare l'integrale di una funzione valutandola numericamente in numero finito $N$ di punti.
Il lettore conoscerà già il metodo dei trapezi e di Simpson (parabole),
che si generalizzano ad approssimare con polinomi di grado $m-1$ (\emph{$m$-sampling}).
Il metodo Monte Carlo più semplice per fare questo calcolo
è valutare la funzione su una sequenza pseudocasuale con distribuzione uniforme
e moltiplicare la media aritmetica dei valori per la lunghezza dell'intervallo.
Infatti per definizione l'integrale della funzione è il valore di aspettazione su una variabile uniforme:
\begin{align*}
	\int_a^b \de x\,f(x)
	&= (b-a)\int_a^b \frac{\de x\,f(x)}{b-a} = \\
	&= (b-a) E[f(x)], \quad p(x) = \frac{\chi_{(a,b)}(x)}{b-a}.
\end{align*}
La stima dell'errore del risultato è la deviazione standard campione sulla media,
quindi ha andamento asintotico $1/\sqrt N$.
Si può dimostrare che l'errore dell'$m$-sampling ha andamento $1/N^{2(m-1)}$,
quindi sembra che non abbiamo ottenuto nulla con il Monte Carlo.
Tuttavia, se l'integrale è in $d$ dimensioni, l'$m$-sampling ha un andamento $1/N^{2(m-1)/d}$,
cioè l'$N$ necessario per ottenere una certa precisione va come $e^{\lambda d}$.
Quindi aumentando il numero di dimensioni diventa rapidamente impossibile in pratica usare l'$m$-sampling.
Inoltre, supponiamo di aver eseguito un integrale e di non essere soddisfatti della precisione:
con l'$m$-sampling bisogna riprocedere da capo aumentando $N$,
mentre con Monte Carlo si può concatenare una nuova sequenza pseudocasuale e riutilizzare i calcoli precedenti.

È comodo non dover scrivere esplicitamente un'intera sequenza pseudocasuale per usarla.
Esistono algoritmi ciclici che generano buone sequenze pseudocasuali molto più lunghe dell'algoritmo.
Tipicamente si studiano algoritmi che generano distribuzioni uniformi e poi si ricavano le altri distribuzioni a partire da quella uniforme con opportune statistiche.
Ad esempio un modo stupido di ottenere una variabile gaussiana è sommarne tante uniformi.

Esiste una statistica definita tutte per le distribuzioni a supporto compatto,
detta \emph{metodo di Von Neumann},
che manda una sequenza di variabili uniformi in una sequenza (di lunghezza casuale e minore) con la distribuzione desiderata:
sia $x$ uniforme con lo stesso supporto della distribuzione obiettivo~$p$
e $y$ uniforme in $(0,\max p)$, \emph{accetto} $x$ se $y<p(x)$ e la \emph{rigetto} altrimenti.
La sequenza delle $x$ accettate è distribuita secondo $p$.

Un altro metodo ancora più generico è il \emph{Markov Chain Monte Carlo} (MCMC).
Una catena di Markov è una sequenza in cui ogni elemento si ottiene a partire da quello precedente.
A causa di questa dipendenza,
in una catena di Markov pseudocasuale gli elementi sono correlati,
quindi la catena va usata sempre tutta insieme e senza fare riferimento all'ordine degli elementi.

L'esempio più semplice di MCMC è \emph{Metropolis}.
Consideriamo una \emph{proposal distribution} $\operatorname{PD}(x;y)$ simmetrica,
cioè $\operatorname{PD}(x;y) = \operatorname{PD}(y;x)$,
con lo stesso supporto della distribuzione obiettivo $p_0$.
Fissiamo l'elemento iniziale della catena $x_0$.
Otteniamo $x_{i+1}$ da $x_i$ estraendo $x$ da $\operatorname{PD}(x;x_i)$ e $u$ uniforme in $(0,1)$ finché non si verifica
\begin{equation*}
	\frac{p_0(x)}{p_0(x_i)} > u.
\end{equation*}
La probabilità di passare da $x_A$ a $x_B$ in un'iterazione è
\begin{equation*}
	p(x_A\rightarrow x_B)
	= \operatorname{PD}(x_B;x_A) \cdot \min\left\{1,\frac{p_0(x_B)}{p_0(x_A)}\right\},
\end{equation*}
quindi il rapporto tra le probabilità nei due versi è
\begin{equation*}
	\frac {p(x_A\rightarrow x_B)} {p(x_B\rightarrow x_A)}
	= \frac{p_0(x_B)}{p_0(x_A)}.
\end{equation*}
Allora si ha una situazione stazionaria stabile quando,
detta $f$ la densità mesoscopica dei punti nella catena,
$f(x)\propto p_0(x)$.
Quindi la catena è una buona sequenza pseudocasuale solo asintoticamente.
Notiamo che nell'algoritmo $p_0$ è usato a meno di una costante moltiplicativa;
questo è molto utile quando si usa il teorema di Bayes: basta scrivere
\begin{equation*}
	p(x|y) \propto p(y|x) \cdot p(x)
\end{equation*}
senza calcolare
\begin{equation*}
	p(y) = \int \de x\, p(y|x)\cdot p(x).
\end{equation*}

% la parte sulla funzione psicometrica la salto perché è tutta inclusa nella lezione di Francavilla
