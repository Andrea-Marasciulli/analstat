% Giacomo Petrillo
% lezione di Morello

\begin{exercise}
	\label{teo:sommadadi}
	Lanciati due dadi coperti, sapendo che la somma esce~6, calcolare la probabilità che scoprendone uno esca~3.
\end{exercise}

\begin{solution}
	Calcoliamo la probabilità condizionata. La probabilità che la somma sia 6 e che un fissato dado sia 3 è $1/36$ (combinazione $(3,3)$), la probabilità che la somma sia 6 è $5/36$ (combinazioni da $(1,5)$ a $(5,1)$), il rapporto è dunque $1/5$.
\end{solution}

\begin{exercise}
	Siano $A$, $B$, $C$ indipendenti a due a due. Vale $P(A\cap B\cap C)=P(A)P(B)P(C)$?
\end{exercise}

\begin{solution}
	No. Controesempio: prendiamo l'insieme universo
	$S = \{1,\dots,9\}$
	e gli insiemi
	$A = \{1, 2, 3\}$,
	$B = \{1, 4, 5\}$,
	$C = \{1, 6, 7\}$.
	Assegnamo come probabilità la cardinalità normalizzata degli insiemi.
	Risulta $P(A)=P(B)=P(C)=1/3$, $P(A\cap B)=\dots=1/9$, ma $P(A\cap B\cap C)=1/9\neq1/27$.
\end{solution}

\begin{theorem}
	Sia $\set{H_i}$ una partizione dell'insieme universo,
	allora qualunque probabilità si scompone in probabilità condizionate rispetto alla partizione:
	\begin{equation*}
		S=\bigcup_iH_i,\ H_i\cap H_j=\emptyset\implies P(A)=\sum_iP(A|H_i)P(H_i).
	\end{equation*}
\end{theorem}

\begin{proof}
	$\begin{aligned}[t]
		A &= A\cap S = A\cap \bigcup_iH_i = \bigcup_i(A\cap H_i) \implies \\
		\implies P(A) &= P\left(\bigcup_i(A\cap H_i)\right) = \sum_i P(A\cap H_i) = \\
		&= \sum_i P(A|H_i)P(H_i).
	\end{aligned}$\vspace{-1em}\\
\end{proof}

% il teorema di Bayes non è qui, probabilmente è nella prima lezione che non ho scritto.

\begin{theorem}[di Bayes]
	Dalla definizione di probabilità condizionata, segue immediatamente che
	\begin{equation*}
		P(B|A) = \frac{P(A|B) P(B)}{P(A)}.
	\end{equation*}
\end{theorem}
