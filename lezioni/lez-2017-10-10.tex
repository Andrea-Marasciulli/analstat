% Giacomo Petrillo
% lezione di Punzi

Consideriamo la media aritmetica di $n$ variabili identiche:
\begin{equation*}
	\bar x \is \frac1n \sum_{i=1}^n x_i.
\end{equation*}
Se la media e la varianza delle $x_i$ sono $\mu$ e $\sigma^2$,
dalle proprietà del valore atteso si ricava immediatamente che
\begin{equation*}
	E[\bar x] = \mu, \quad \var[\bar x] = \frac{\sigma^2}{n}.
\end{equation*}



\chapter{Inferenza}